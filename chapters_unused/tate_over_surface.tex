\chapter{Airy structures and complex surfaces}

    Let \(X\) be a complex surface with divisor \(R\), \(\omega\) be a symplectic form on \(X\). This chapter examines a particular example of an Airy structure.

    \subsection{Tate spaces and curves on a complex surface}
    
    We can now apply the previous ideas to construct a bundle of Tate spaces over a complex algebraic surface. As in \cite{ks_airy}, define a symplectic (Tate) vector space, consisting of differentials, as follows:
    \[ W_{\text{Airy}}=\left\{ J=\sum_{n\in\mathbb{Z}} J_n \, z^{-n}\frac{dz}{z}\mid J_0=0, \exists N \text{ such that }J_n=0,\text{for }n>N\right\}.
    \]
    There is a symplectic form given by the residue pairing:
    \[ \Omega_{W_\text{Airy}}(df, \eta)=\Res_{z=0}f \eta,\quad df, \eta \in W_\text{Airy}.
    \]
    Now given \(\Sigma \subset X\), a Weil divisor \(R\subset\Sigma\), and local coordinates \(z_\alpha\) defined in a neighbourhood of \(\alpha\in R\), define \[W_\Sigma =(W_{\text{Airy}})^R.\]
    We define a polarisation, so a choice of \(V_\Sigma\) such that \(W_\Sigma = V_\Sigma \oplus V_\Sigma^*\) in \ref{}.
    
    A main result of this chapter is to show that the tensor \(A_\Sigma\) maps to a tensor \(\overline{A}_\Sigma\) also known as the \emph{Donagi-Markman cubic}, under the quotient described generically in the previous chapter.
    
    \begin{thm}  \label{main}
    The image of the tensor $A_\Sigma$ under the quotient map $V_\Sigma\to\overline{V}_\Sigma$  is the tensor $\overline{A}_\Sigma$.
    \begin{align}  \label{tensquot}
    V_\Sigma\otimes V_\Sigma\otimes V_\Sigma&\to\overline{V}_\Sigma\otimes\overline{V}_\Sigma\otimes\overline{V}_\Sigma\\
    A_\Sigma&\mapsto \overline{A}_\Sigma.\nonumber
    \end{align}
    \end{thm}

    
    
    
    
    \newpage 
    
    \subsection{}


    Let \(X\) be a complex surface with divisor \(R\), \(\omega\) be a symplectic form on \(X\), and now let \(\mathcal{F}\) be a Lagrangian foliation, and a local involution \( \sigma_{\alpha}\) for \( \alpha \in R\).
    
    Let  $(X,\omega)$ be a symplectic surface with a Lagrangian foliation $\mathcal{F}$. 
    
    Consider an infinite dimensional space $Coord$ parameterizing the space of tuples $(s,x,y)$, where $s \in X$ and $(x,y)$ is a formal coordinate system at the point $s$ such that the symplectic form $\omega$ is equal to $dx \wedge dy$ and the tangent bundle of the foliation $\mathcal{F}$ is generated by the vector field $\partial_y$. 
    There is a $3:1$ degree map $\pi:Coord \rightarrow Discs$ which sends a point $(s,x,y)$ to the formal disc with coordinate $z$ embedded via the assignment $z \rightarrow (x=z^2,y=z)$. 
    
    The pull-back of the vector bundle $\mathcal{H}_t$ defined in Section \ref{formal} is a canonically trivialized symplectic vector bundle, whose fibers are identified with the space $W = \otimes_{r \in R} W_{Airy} $, where
    $$W_{Airy}=\mathrm{Ker}(\mathrm{Res}_z: \mathbb{C} \lParen z \rParen) dz \rightarrow \mathcal{C})$$
    and $R$ denotes a divisor.
    
    
    Denote by $V^*$ the tangent space of the Lagrangian $\mathcal{L}$ at the origin and by $V$ its topological dual. This gives the decomposition $W=V \oplus V^*$.
    By \cite[Corollary 7.2.1]{KSoAir} the vector space $V$ carries an Airy structure. 
    
    %Therefore, the Lagrangian $\mathcal{L}$ defines a cyclic deformation quantization module which is generated by wave function $\psi=\exp \left( \sum_{g \geq 0} \hb^{g-1}\, S_g  \right)$.
    
    
    %In the $k$-th formal neighbourhood of $0\in W$, for each $k$, a quadratic Lagrangian $\cl$ is defined as follows.  An element $\eta$ in the $k$-th formal neighbourhood of $0\in W$ lives in $\cl$, i.e. it is annihilated by the ideal defining $\cl$, if it satisfies the following residue constraints \cite{KSoAir}.  
    
    Consider a curve \( \Sigma_0\) on \(X\). Meromorphic differentials on \( \Sigma_0 \) are identified with a space \(V\).
    
    Let $(x_\alpha,y_\alpha)$ be local coordinates defined in a neighbourhood of $\alpha\in R\subset\Sigma_0\subset X$.
    These coordinates are chosen such that $x_\alpha-y_\alpha^2=0$ parameterises $\Sigma_0$:
    \begin{align}
    &\Res_{\alpha} \left(y_{\alpha}-\frac{\eta}{dx_{\alpha}}\right)x_{\alpha}^m \, dx_{\alpha} =0,  \quad m\geq 1,  \\
    &\Res_{\alpha} \left(y_{\alpha}-\frac{\eta}{dx_{\alpha}}\right)^2 x_{\alpha}^m\, dx_{\alpha}=0,  \quad m\geq 0. 
    \end{align} 
    The condition that the differential $\Res_\alpha\eta x_\alpha^mdx_\alpha=0$ for $m\geq 0$ is equivalent to $\eta$ having skew-invariant principal part under the local involution $\sigma_\alpha$.\\



    The system of equations:
    \[\mathrm{Res}_z(x^k ydx)=0, k \geq 1 \qquad \mathrm{Res}_z(x^k y^2 dx)=0, k \geq 0,\]
    gives the formal germ of a Lagrangian \(\mathcal{L}\) of a Tate space $W$.
