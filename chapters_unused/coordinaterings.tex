    \section{Coordinate rings on a vector space}
    
    The symmetric algebra functor is useful for doing coordinate free algebraic geometry over a vector space. For the category of vector spaces, the dual functor gives the dual vector space.  Taking the dual is contravariant, so reverses directions of arrows. 
    
    So it is natural to consider a presheaf on a vector space given by the functor \( F = \rS \cdot (\, )^*\), instead of just considering the symmetric algebra.
    
    

    
    %As every vector topology is translation invariant, the map \( V \rightarrow  V\) defined by \( v \rightarrow v + p\), is a homeomorphism, it is sufficient to construct the sheaf o
    %It is sufficient to construct the sheaf structure over linear subspaces.
    
    
    Given a subset \(U\) (not necessarily a linear subspace, but the argument goes through in terms of basis elements), assign it the value of \( \rS(U^*)\). 
    Now for subsets \(P \subset Q \subset V \), there is an inclusion morphism \( \iota : P \rightarrow Q \).  Taking duals this gives a morphism \( \iota^* : Q^* \rightarrow P^*\). By universality this gives a restriction morphism, \( \mathrm{res}_{PQ} = \rS(\iota^*)  : \rS(Q^*) \rightarrow \rS(P^*)\). Likewise by universality, \( \mathrm{id} : U \rightarrow U \) lifts to a morphism \( \rS(\mathrm{id}) = \mathrm{id}\).
    This also has the necessary functorial properties. Given \( P \rightarrow Q \rightarrow R\), this lifts to a composition of restriction morphisms as expected. 
    
    
    
    Finally to show that this does indeed generate a sheaf, 
    %\( \Gamma(V, \mathcal{O}_V) = S(V^*)\).
    %Given a sub vector space \(U \subset V\), which is a morphism \( U \rightarrow V\), by the universal property of the symmetric algebra, this induces a restriction map \( \mathcal{O}(V) \rightarrow \mathcal{O}(U)\)
    
    So for a finite symplectic vector space \(W \cong V \otimes V^*\), 
    picking a basis \( \mathcal{B} = \{ e_i\}\) for \(V\) by the previous construction, this is sent to the symmetric algebra \( \mathbf{k}[\mathcal{B}^*]\), where the \( \mathcal{B}^*\) are now a set of coordinates on \(V\).  Rephrasing this, these are \emph{Darboux} coordinates. Given linear coordinates \(\{x^k\mid k\in I\} \) on \(V^*\), then \( x^k \in (V^*)^* \cong V\). Likewise pick coordinates \(\{y_k\mid k\in I\} \) on \(V\), so that \(y_k \in  V^*\). This gives a ring 
    \( \mathbf{k}[W] \cong \mathbf{k}[x^\bullet] \otimes_{\mathbf{k}} \mathbf{k} [y^\bullet]\), where the bullets represent all the \(x^k\) and \(y_k\). 
    
    
    Overall this does give a candidate for a locally ringed space 
    \( (W, \mathcal{O}_W)\), where \( \Gamma(W, \mathcal{O}_W ) = \mathbf{k}[W]\). However, if \(W\) was infinite dimensional this construction of \( \mathcal{O}_W\) would forget information about the topology on \(V^*\). The symmetric algebra as defined with algebraic tensor products does not capture all the information. 
    
    The functor mapping from the category of topological vector spaces to commutative algebras is foregtful, so this suggests two approaches.
    
    One option is to embed \(W\) in an ind-scheme.
    The alternative approach is replacing the tensor product with a topological complete tensor product, and looking for a functor to the category of topological algebras.
    
    \subsection{Completed symmetric algebras}
    