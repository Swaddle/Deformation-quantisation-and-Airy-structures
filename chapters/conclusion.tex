\chapter{Future speculation}

There are several areas which should be investigated as a result of this thesis. The first is the global or perhaps non perturbative wavefunction like object defined from deformation quantisation. In chapter (\ref{chapter:deformation}) deformation quantisation  gives a star product equation 
\[ H \star w = 0,\]
where \(w\) is some function in \( \mathcal{O}_X \lBrack \hslash \rBrack \). Instead of trying to solve this equation, we looked at a Weyl-algebra representation and then locally solved an operator equation with the WKB method:
\[ \widehat{ H } \exp(S)  = 0.\]
It would be interesting to see if \(w\) exists as some global or even algebraic object, or if it could be patched together from local wavefunctions \( \exp(S)\).

Another area concerning deformation quantisation is the fact that there are multiple deformation quantisation modules which quotient to the same classical object. For example setting other constants for \(J\) in chapter (\ref{chapter:deformation}). In this thesis we often implicitly worked with whatever choice of \(J\) would give the simplest equations. Different choices of \(J\) may be interesting in applications like quantum curves.

Another direction would be to investigate the \(F_{g}\) from the deformation quantisation of \( \mathcal{B}\), realised as an intersection in an extension. We outlined a practical way to compute higher order terms in \(F_0\) with fixed point iteration. Practically computing more terms in \(F_0\) with this procedure could be done on a computer.

Similarly, \( F_g\) could be derived from the WKB expansion of the quantisation of the equations describing \( \mathcal{B}\) in the intersection. Alternatively, the path like integral for \( \psi_{\mathcal{B}}\) could be investigated in more depth. Writing out Feynman like rules would be an interesting problem.

It would also be interesting to understand the integral in more depth. Although it looks like a Gaussian on the bigger space, it qualitatively looks more like a Laplace transform. Also, although it is integrating over infinitely many variables, it only depends on finitely many of the variables defined on the intersection, or on the space described by the symplectic reduction. A similar situation arises in physics.